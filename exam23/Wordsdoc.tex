\documentclass[14pt]{article}

\usepackage{amsmath}
\usepackage{amssymb}
\usepackage[a4paper, margin=2cm]{geometry}
\usepackage[russian]{babel}

\title{Билеты по физике}
\author{1 курс, 2 семестр}

\begin{document}
    %% Собрал титульник
    \maketitle
    %%\tableofcontents
    %% Впихал секции





    %%   2222
    %% 22  22
    %%     22
    %%     22
    %%     22
    %%222222222222
    \section{Кинетическая энергия тела, вращающегося вокруг неподвижной оси. Плоское движение}
    Скалярная физическая величина, являющаяся мерой движения материальных точек, образующих рассматриваемую механическую систему, и зависящая только от масс и модулей скоростей этих точек
    \newline
    \begin{center}
        \big[$E_k = \frac{mv^2}{2}$] 
    \end{center}
    Но во время вращения вокруг неподвижной оси
    \begin{center}
        $E_{к} = \frac{1}{2} I \omega^2$
    \end{center}
    $I$ — Момент инерции тела относительно оси вращения,
    $\omega$ -- Угловая скорость.

    \textit{Что такое момент инерции относительно оси вращения?}
    \newline — это физическая величина, которая характеризует сопротивление тела изменению его состояния вращения вокруг этой оси. 
    \newline \textbf{Проще говоря}, он показывает, насколько трудно повернуть или остановить тело, вращающееся вокруг определённой оси. 
    \[ I = \sum m_i r_i^2 \]  
    \( m_i \) — масса каждого элементарного участка тела,
    \( r_i \) — расстояние этого участка до оси вращения. 
    \textbf{Чем больше момент инерции, тем сложнее телу изменять свою угловую скорость.} 
    Он зависит от распределения массы относительно оси вращения: чем дальше масса расположена от оси, тем больше будет момент инерции.
    \textbf{Плоское движение} - это движение тела в двумерной плоскости. В таком движении тело меняет свою позицию в плоскости за счет изменения координат с течением времени. 
    \newline \textbf{Основные параметры:} 
    \newline Координаты \((x(t), y(t))\), задающие положение тела в любой момент времени. 
    \newline Путь — длина траектории, по которой движется тело. -
    \newline Скорость — вектор \(\vec{v}(t) = \frac{d\vec{r}}{dt}\), где \(\vec{r}(t) = x(t)\hat{i} + y(t)\hat{j}\). 
    \newline Ускорение — вектор \(\vec{a}(t) = \frac{d\vec{v}}{dt}\). 
    \newline \textit{ Примеры:} движение по окружности, бросок тела под углом, движение по параболе и т.д.


    %%   2222
    %% 22    22
    %%     22
    %%   22
    %% 22222222
    \section{Молекулярная форма движения. Силы взаимодействия между молекулами. Агрегатные состояния вещества. Идеальный газ.}
    \textbf{Когда все молекулы вещества двигаются "в одном" направлении одновременно}
    \newline На молекулярном уровне частицы вещества  находятся в постоянном движении. В зависимости от температуры и агрегатного состояния, это движение может быть разным: 
    \newline \textit{В газах}  — молекулы движутся беспорядочно, со скоростями, примерно соответствующими температуре. Их движение хаотичное, быстрые столкновения. 
    \newline \textit{ В жидкостях} — молекулы тоже движутся, но ближе друг к другу и с меньшей скоростью, что позволяет им скользить друг по другу. -
    \newline \textit{В твердых телах}  — молекулы колеблются около своих положений, образуя регулярную кристаллическую или аморфную структуру.
    
    \textbf{Силы взаимодействия между молекулами} — это силы, которые действуют между соседними молекулами и определяют свойства веществ.
     Основные виды таких сил:
    \newline \textit{1.Силы Ван-дер-Ваальса} - дисперсионные или слабые силовые взаимодействия: возникают из-за временных поляризаций электронных облаков в нейтральных молекулах. Они относительно слабы, но важны для жидкостей и твердых тел. 
    \newline \textit{2.Дипольные силы} действуют между полярными молекулами, у которых есть постоянный электрический диполь. Эти силы сильнее, чем силы Ван-дер-Ваальса, и влияют на температуру кипения и растворимость. 
    \newline \textit{3. Ионные силы } электростатические взаимодействия между ионами в ионных соединениях (например, соли). Они очень сильные и отвечают за структуру кристаллов и высокие температуры плавления.
    Эти силы определяют агрегатное состояние вещества, его твердость, температуру плавления и кипения, растворимость и другие свойства.

    \textbf{Агрегатные состояния вещества}
    \newline \textit{Всего есть 4 агрегатные состояния вещества}
    \newline \textbf{\textit{Твёрдое вещество}} - молекулы вещества находятся в форме кристалической решётки, молекулы "колеблятся" около своих стационарных позиций в решётке
    \newline \textbf{\textit{Жидкость}} - молекулы вещества находятся в движении, находятся друг относительно друга рядом. Сталкиваются,  могут перемещаться от одного конца вещества к другому
    \newline \textbf{\textit{Газ}} - Молекулы вещества находятся в движении, находятся далеко друг относительно друга двигаются быстро и сталкиваются.
    \newline \textbf{\textit{Плазма}} - В плазме молекулы ведут себя очень активно и нестабильно. Из-за высокой энергии и температуры многие молекулы ионизируются, то есть теряют электроны и превращаются в ионы и свободные электроны. Также происходят столкновения между частицами: электронами, ионами и оставшимися нейтральными молекулами, вызывая возбуждение, распад или рекомбинацию молекул. В результате в плазме преобладают ионы, электроны и возбужденные частицы, а сами молекулы обычно распадаются или переходят в ионизированное состояние.

    \textbf{Идеальный газ}
    \newline \textit{Модель}, в которой предполагается, что молекулы или атомы не взаимодействуют друг с другом, за кроме случайных столкновений. Также считается, что молекулы имеют очень малый размер по сравнению с расстояниями между ними. 
    \newline \textit{В таком газе молекулы ведут себя следующим образом:}  
    1 - Они движутся случайно и со скоростями, соответствующими температуре. 
    2 - Столкновения между молекулами кратковременны и абсолютно упругие, то есть при столкновениях сохраняется энергия. 
    3 - Молекулы не притягиваются и не отталкиваются друг от друга, кроме столкновений.
    4 - Вся энергия молекул — это их кинетическая энергия, связанная с их движением.
    \textbf{ Такая модель помогает понять основные законы поведения газов.}
    \begin{center}
        \big[$PV=\nu RT$\big] - Уравнение идеального газа.
    \end{center} 

    %% 3333
    %%    33
    %%  33
    %%    33
    %% 3333
    \section{Размеры и масса молекул. Число Авогадро. Масса моля}
    \textbf{Размеры и масса молекул}
    Размер молекул варьируется в зависимости от типа вещества, но в среднем он составляет порядка 0,2 — 0,5 нанометров (нм), то есть 2 — 5 ангстремов . 
    \newline Для сравнения: - Вода: молекула воды имеет диаметр примерно 0,27 нм. - Атом водорода: около 0,1 нм. - Ароматические углеводороды или молекулы сложных органических соединений могут быть чуть больше — до 1 нм.
    
    \textbf{Число Авогадро}
    - Это число, которое отражает количество молекул в которая показывает, сколько частиц (молекул, атомов или ионов) содержится в одном моле вещества. 
    Оно обозначается символом \( N_A \). 
    Значение числа Авогадро примерно равно: \[ N_A \approx 6,022 \times 10^{23} \] Это значит, что в одном моле вещества содержится примерно 602 квадриллиона (602 с 21 нулём) частиц. Это число помогает связать макроскопические измерения (например, массу вещества) с микроскопическими свойствами (числом частиц). Число Авогадро было установлено экспериментально и является одним из фундаментальных постоянных природы.
     
    \textbf{Масса поля}
    Термин «масса поля» обычно относится к понятию в теории поля, где оно описывает величину, связанную с массой переносчика взаимодействия (например, у фотона масса равна нулю, а у W и Z бозонов — есть). В более общем смысле, масса поля связана с его энергетическими свойствами и влияет на его поведение и взаимодействия.

    
    %% 44  44
    %% 44  44
    %% 444444
    %%     44
    %%     44
    \section{Давление и температура. Термодинамическая шкала температур}
    \textbf{\textit{Давление и температура}}
    - Давление [ $P$ ] - В молекулярно-кинетической теории (МКТ) температура связана с средней кинетической энергией молекул: \[ \frac{3}{2} k_B T = \text{средняя кинетическая энергия одной молекулы} \] где \(k_B\) — постоянная Больцмана. Давление в газе определяется частотой и силой столкновений молекул со стенками сосуда: \[ pV = N k_B T \] или, в виде уравнения для концентрации \(n = N/V\): \[ p = n k_B T \] Это означает, что при постоянной концентрации давление прямо пропорционально температуре.
    
    \textbf{\textit{Термодинамическая шкала температур}}
    - Основана на связи между температурой и средней кинетической энергией молекул. Она определяется так: при данной температуре средняя кинетическая энергия молекулы пропорциональна абсолютной температуре \( T \), измеряемой в шкале Кельвина (К). Абсолютный ноль (0 К) — это температура, при которой средняя кинетическая энергия молекул равна нулю, то есть молекулы «заморожены» и не совершают тепловых движений. Эта шкала служит основой для абсолютной температуры, где температура прямо связана с внутренней энергией системы.

    \section{Методы исследования молекулярной формы движения}
    \textit{Бывают спектроскопические, диффузионные и структурные.} 
    
    \textbf{Ключевые из них:} - 
    \newline \textbf{\textit{Инфракрасная и рамановская спектроскопия}} — позволяют изучать колебания и вибрации молекул, что дает информацию о форме и симметрии. 
    \newline \textbf{\textit{Ядерный магнитный резонанс}}  — исследует вращательные движения, локальную динамику и конфигурацию молекул в жидких средах.
    \newline \textbf{\textit{Рассеяние света и нейтронов}}  — методы динамического рассеяния позволяют определить размеры, формы молекул, а также их трансляционные и вращательные движения.
    \newline \textbf{\textit{Молекулярное моделирование и компьютерное моделирование}}  — позволяют визуализировать и анализировать формы, вращения и колебания молекул. 
    \newline \textbf{\textit{Кристаллографические методы (рентгенография)}}  — дают трехмерную структуру твердых кристаллов, что помогает понять их молекулярную форму и возможные движения в твердом состоянии.

    \section{Микроскопическое и макроскопическое состояние термодинамической системы. Первый постулат термодинамики. Экспериментальные законы идеальных газов}
    \textbf{Микроскопическое и макроскопическое состояние термодинамической системы}
    \newline - Микроскопическое состояние термодинамической системы характеризует точное расположение, скорости и энергии каждой частицы системы — это конкретное конфигурационное состояние системы на уровне отдельных молекул или атомов. Макроскопическое состояние — это совокупность средних характеристик системы, таких как температура, давление, объем, которые можно измерить и описать без учета деталей микроскопического уровня. Оно определяется статистическими свойствами множества микросостояний и служит для практического описания и анализа поведения системы в целом.
    
    \textbf{Первый постулат термодинамики}
    \newline - изменение внутренней энергии системы равно сумме переданного тепла и выполненной работы: \[ \Delta U = Q + W \] где: - \(\Delta U\) — изменение внутренней энергии, - \(Q\) — количество тепла, переданного системе, 
    \newline \(W\) — работа, выполненная системой.
    \newline Это означает, что \textit{энергия в замкнутой системе не исчезает и не появляется из ниоткуда}  — она только передается или преобразуется.

    \section{Уравнение состояния идеального газа. Уравнения Менделеева-Клапейрона и Клаузиуса.}
    \textbf{\textit{Уравнение состояния идеального газа}}
    \begin{center}
        \big[$PV=\nu RT$\big]
    \end{center}
    \textbf{\textit{Уравнения Менделеева-Клапейрона и Клаузиуса}}
    \begin{center}
        \[ PV = \nu RT \] \(P\) — давление, \(V\) — объем, \(\nu\) — количество молей, \(R\) — газовая постоянная, \(T\) — температура в Кельвинах
    \end{center}
    \textbf{\textit{Уравнение Клаузиуса}}
       \[ \oint \frac{\delta Q}{T} \leq 0 \] Это говорит о том, что для циклов в замкнутой системе интеграл по теплу, делённому на температуру, не может быть положительным. В простом виде оно показывает, что невозможно полностью преобразовать тепло в работу без потерь.

    \section{Основное уравнение молекулярно-кинетической теории газов.}
    \[ PV = \frac{1}{3} N m \overline{v^2} \] где: - \(P\) — давление, - \(V\) — объем, - \(N\) — число молекул, - \(m\) — масса одной молекулы, - \(\overline{v^2}\) — среднеквадратичное значение скорости молекул. 
    \newline Это уравнение показывает, что давление создается столкновениями молекул с стенками сосуда, и связано с их средней скоростью. Из этого уравнения выводится также формула для средней кинетической энергии молекулы: \[ \overline{E_k} = \frac{3}{2} k_B T \] где \(k_B\) — постоянная Больцмана, а \(T\) — температура.
    
    \section{Основное уравнение молекулярно-кинетической теории газов через кинетическую энергию молекул. Температура как мера средней кинетической энергии поступательного движения молекул}
    \[ PV = \frac{2}{3} N \overline{E_k} \] или в виде средней кинетической энергии одной молекулы: \[ \overline{E_k} = \frac{3}{2} k_B T \] 
    \newline где: - \(P\) — давление, - \(V\) — объем, - \(N\) — число молекул, - \(\overline{E_k}\) — средняя кинетическая энергия одной молекулы, - \(k_B\) — постоянная Больцмана, - \(T\) — температура в Кельвинах.
    \newline Это уравнение показывает, что давление газа обусловлено столкновениями молекул с стенками сосуда и прямо связано с их средней кинетической энергией. Температура — это мера средней кинетической энергии поступательного движения молекул: яя кинетическая энергия и, следовательно, большее давление при постоянном объеме и числе молекул.


    \section{Работа, совершаемая газом при изменении объема}
    \[ A = \int_{V_1}^{V_2} P\, dV \] Для идеального газа, если давление зависит от объема и температуры, эта работа может быть выражена конкретной формулой в зависимости от типа процесса. Например: - при изохорном (постоянном объеме) процессе работа равна нулю, так как объем не меняется; - при изобарном (постоянное давление) процессе: \[ A = P (V_2 - V_1) \] - при изотермическом процессе (постоянная температура): \[ A = nRT \ln \frac{V_2}{V_1} \] где \(n\) — число молей, \(R\) — газовая постоянная, \(T\) — температура.


    \section{Первое начало термодинамики}
    изменение внутренней энергии системы равно разности между количеством теплоты, переданной системе, и работой, совершенной системой: \[ \Delta U = Q - A \] или, иногда, в форме: \[ \Delta U = Q + W \] где: - \(\Delta U\) — изменение внутренней энергии, - \(Q\) — теплота, полученная системой, - \(A\) или \(W\) — работа, выполненная системой. 
    \newline Это выражение отражает закон сохранения энергии для термодинамических процессов.


    \section{Теплоемкость тел. Внутренняя энергия и теплоемкость идеального газа.}
    \textbf{\textit{Теплоёмкость тел}}— это мера того, сколько тепла нужно передать телу, чтобы повысить его температуру на 1 единицу тепла (например Кельвин). 
    \newline Обозначается обычно \( C \) и выражается формулой: 
    \begin{center}
        \big[\[ C = \frac{Q}{\Delta T} \]\big]
        где \( Q \) — переданное тепло, а \( \Delta T \) — изменение температуры
    \end{center}
    \textit{Существует несколько видов теплоемкости}:
    \newline \textbf{Общая теплоемкость} — для всего тела, зависит от его массы и вида вещества. 
    \newline \textbf{Удельная теплоемкость} (\( c \)) — теплоемкость на единицу массы, выражается как \( c = \frac{C}{m} \). 
    \newline \textbf{Молярная теплоемкость} (\( C_m \)) — теплоемкость на один моль вещества. Для тела массой \( m \): \[ Q = mc \Delta T \] Это означает, что чтобы повысить температуру тела на \( \Delta T \), нужно передать тепло \( Q \), пропорциональное массе и удельной теплоемкости.
    
    \textbf{\textit{Внутренняя энергия и теплоёмкость идеального газа}}
    - \textit{зависит только от температуры и связана с движением молекул}.
     Для одного моля газа она выражается как: 
     \begin{center}
        \big[\[ U = \frac{f}{2} R T \]\big]
        где: - \( f \) — число степеней свободы молекул (например, для моноатомного — 3, для диатомного — 5), - \( R \) — газовая постоянная, - \( T \) — абсолютная температура
     \end{center}
    \textit{Теплоемкость идеального газа связана с изменением внутренней энергии при изменении температуры}: - 
    
    \textbf{Молярная теплоемкость при постоянном объеме}  
    \begin{center}
        \big[(\( C_v \))**: \[ C_v = \left(\frac{\partial U}{\partial T}\right)_V = \frac{f}{2} R \]\big]
    \end{center}
    
    \textbf{Молярная теплоемкость при постоянном давлении}
    \begin{center}
        \big[(\( C_p \))**: \[ C_p = C_v + R = \left(\frac{f}{2} + 1\right) R \]\big]

    \end{center}

    \textit{Эти величины показывают, сколько джоулей тепла нужно добавить или убрать, чтобы изменить температуру газа на 1 К, при постоянных объеме или давлении}.

    %%  111    3333
    %%11 11       33
    %%   11    333
    %%   11       33
    %% 111111  3333 
    \section{Молярные теплоемкости идеальных газов при изохорическом и изобарическом процессах. Уравнение Роберта-Майера}
    \textbf{Молярная теплоемкость при постоянном объеме}
    \[C_v = \left( \frac{\partial U}{\partial T} \right)_V\]
    \textit{Для идеального газа}
    \begin{center}
        \big[\[C_v = \frac{f}{2} R\]\big]
        где \(f\) — число степеней свободы молекулы.
    \end{center}
    
\textbf{Молярная теплоемкость при постоянном давлении}

\[C_p = C_v + R = \left( \frac{f}{2} + 1 \right) R\]


\textbf{Уравнение Роберта-Майера}
%% Уравнение Роберта-Майера связывает внутреннюю энергию и работу при изменениях в состоянии газа.
\textit{Это уравнение связывает изменение внутренней энергии и работу, выполненную газом при изменении температуры и объема. В общем виде оно выглядит так:}
\begin{center}
    \big[\[dQ = dU + PdV\]\big]
    
где \(dQ\) — количество переданного тепла, \(dU\) — изменение внутренней энергии, \(P\) — давление, \(dV\) — изменение объема.
\end{center}

\textit{Для идеального газа} 
с \(U = C_v T\) и \(PV = RT\):
\[dQ = C_v dT + P dV\]

\textit{или} \(P dV = R dT\):
\[dQ = (C_v + R) dT = C_p dT\]
\newline \textit{Это означает, что при постоянном давлении теплота, переданная газу, пропорциональна изменению температуры с коэффициентом \( C_p \)}

Итак:  
\( C_v = \frac{f}{2} R \),
\( C_p = \left( \frac{f}{2} + 1 \right) R \),


    
\section{Адиабатический процесс. Уравнение Пуассона}
\textbf{\textit{Адиабатический процесс}} - Это термодинамический процесс, при котором теплообмен с окружающей средой отсутствует (\(Q=0\)). 
\newline В результате такие процессы характеризуются изменением давления, объема и температуры без передачи тепла.
\newline Для идеального газа в адиабатическом процессе выполняется уравнение Пуассона: 
\begin{center}
    \big[\[ PV^{\gamma} = \text{const} \]\big]
    или в виде, связанной с температурой и давлением:
    \[ TV^{\gamma - 1} = \text{const} \] \[ P^{1-\gamma} T^{\gamma} = \text{const} \]
    где: - \(P\) — давление, - \(V\) — объем, - \(T\) — температура, 
\end{center}
\textbf{Показатель адиабаты (коэффициент адиабатического расширения} 
\(\gamma = \frac{C_p}{C_v}\) 

\textbf{\textit{Уравнение Пуассона}} - Это уравнение описывает связь между давлением и объемом или температурой при адиабатическом процессе для идеального газа:
\begin{center}
    \big[\[ P V^{\gamma} = \text{const} \] или \[ T V^{\gamma - 1} = \text{const} \]\big]
    
    или
    \big[\[ \frac{P_1 V_1^{\gamma}}{P_2 V_2^{\gamma}} = 1 \]\big]
    где индексы 1 и 2 обозначают начальные и конечные состояния.
\end{center}



\section{Элементы теории вероятностей. Функция распределения случайной величины}
- изучает закономерности случайных событий и случайных величин. 

\textbf{Основные понятия включают} 
\newline \textit{Случайное событие} — исход, который может произойти или не произойти при проведении эксперимента. 
\newline \textit{Вероятность события} — числовая характеристика, показывающая степень его вероятности, обозначается \(P(A)\). 
\newline \textit{Случайная величина} — функция, которая каждому исходу эксперимента ставит в соответствие число. 
\newline \textit{Функция распределения случайной величины} Для случайной величины \(X\) функция распределения \(F_X(x)\) задается как: 
\begin{center}
    \big[\[ F_X(x) = P(X \leq x) \]\big]
    Это — функция, которая для каждого числа \(x\) дает вероятность того, что случайная величина \(X\) примет значение, не превышающее \(x\).
\end{center}


\textbf{\textit{Свойства функции распределения}} 
\newline \textit{Неубывание} \(F_X(x)\) — неубывающая функция. 
\newline \textit{Пределы} \(\lim_{x \to -\infty} F_X(x) = 0\), \(\lim_{x \to +\infty} F_X(x) = 1\). 
\newline \textit{Линейность и непрерывность} для дискретных случайных величин \(F_X(x)\) — ступенчатая, для непрерывных — непрерывная функция


\section{Пространство скоростей молекул. Связь между функциями распределения Максвелла по скоростям и компонентам скоростей}

\textbf{\textit{Пространство скоростей молекул}}
\newline — это трехмерное пространство, где каждая точка соответствует возможной скорости молекулы, заданной компонентами \(v_x, v_y, v_z\).
\newline \textit{Функция распределения Максвелла по скоростям выглядит так:}
\begin{center}
    \big[\[ f(\vec{v}) = \left(\frac{m}{2\pi k T}\right)^{3/2} \exp\left(-\frac{m(v_x^2 + v_y^2 + v_z^2)}{2k T}\right) \]\big]
    где \(m\) — масса молекулы, \(k\) — постоянная Больцмана, \(T\) — температура.
\end{center}

\textbf{\textit{Связь между функцией по полной скорости и по компонентам в том, что функция Максвелла по скорости — это произведение трех одинаковых одномерных гауссианов}}
\begin{center}
    \big[\[ f(\vec{v}) = f_x(v_x) \times f_y(v_y) \times f_z(v_z) \] \big]
    где каждый компонент имеет вид:
    \big[\[ f_x(v_x) = \left(\frac{m}{2\pi k T}\right)^{1/2} \exp\left(-\frac{m v_x^2}{2k T}\right) \]\big]
    Аналогично для
    \big[\(f_y(v_y)\) и \(f_z(v_z)\).\big]
\end{center}

\textit{Это показывает, что компоненты скоростей — независимы и распределены по гауссовскому закону, а полное распределение — произведение по компонентам.}


    
\section{Зависимость функции распределения Максвелла по компоненте скорости от температуры}
\begin{center}
   textit{Чем выше температура, тем больше среднеквадратичная скорость и шире распределение компоненты скорости по Максвеллу.}
\end{center}

\textbf{\textit{Рассмотрим на примере одномерной функции}}
\begin{center}
     \(v_x\): \[ f_x(v_x) = \left(\frac{m}{2\pi k T}\right)^{1/2} \exp\left(-\frac{m v_x^2}{2k T}\right) \]
    Здесь: - \(m\) — масса молекулы, - \(k\) — постоянная Больцмана, - \(T\) — абсолютная температура
\end{center}

\textbf{Как меняется функция при изменении температуры:}
\newline Масштаб распределения - компонент перед экспонентой 
\begin{center}
    \(\left(\frac{m}{2\pi k T}\right)^{1/2}\)
    уменьшается при росте температуры, поскольку в знаменателе стоит \(T\). Это означает, что с увеличением температуры ширина распределения возрастает: распределение становится более "плоским" и широким.
\end{center}

\textbf{Стандартное отклонение (среднеквадратичное значение скорости)}
\newline \begin{center}
    \[ \sigma_{v_x} = \sqrt{\frac{k T}{m}} \]
\end{center}

\textit{При росте температуры} \(\sigma_{v_x}\) увеличивается, то есть компонента скорости становится более разбросанной

\textbf{Общий эффект}
\newline Повышение температуры приводит к тому, что компоненты скоростей имеют более широкое распределение, увеличивается среднеквадратичное значение скорости, а молекулы движутся быстрее в среднем



\section{Распределение Максвелла по абсолютным скоростям молекул}
\textbf{\textit{описывает вероятность того, что молекула имеет определённую абсолютную скорость \(v\). Это распределение учитывает все три компоненты скорости и даёт вероятность для полного модуля скорости.}}
\textit{Это распределение показывает, что с ростом температуры увеличивается и средняя, и наиболее вероятная скорость молекул, а вероятностное распределение смещается в сторону более высоких скоростей}
\begin{center}
    \[ f(v) = 4\pi \left(\frac{m}{2\pi k T}\right)^{3/2} v^2 \exp\left(-\frac{m v^2}{2k T}\right) \]
    \newline где: - \(v\) — модуль скорости молекулы, - \(m\) — масса молекулы, - \(k\) — постоянная Больцмана, - \(T\) — абсолютная температура.
\end{center}
\textbf{Ключевые особенности этого распределения}
\newline\textit{Фактор} \(v^2\) - вероятность увеличивается с ростом скорости вначале (по мере увеличения \(v\)), поскольку увеличивается число способов, при которых молекула может иметь такую скорость
\newline \textit{Экспоненциальный спад} -  при очень больших скоростях вероятность быстро убывает, поскольку энергии движений становятся слишком высокими
\newline \textit{Наиболее вероятная скорость \(v_{mp}\)} - её можно найти, приравняв производную \(f(v)\) к нулю, и она равна: \[ v_{mp} = \sqrt{\frac{2kT}{m}} \] 

\textit{Средняя скорость} - $(\langle v \rangle = \sqrt{\frac{8kT}{\pi m}}\)$
\textit{Среднеквадратичная скорость} - $$\sqrt{ \frac{3kT}{m} }$$



\section{Наивероятнейшая, средняя и среднеквадратичная скорости молекул}
\textit{Наиболее вероятная скорость \(v_{mp}\)} - её можно найти, приравняв производную \(f(v)\) к нулю, и она равна: 
\begin{center}
    \big[\[ v_{mp} = \sqrt{\frac{2kT}{m}} \]\big]
\end{center} 
\textit{Средняя скорость} - $(\langle v \rangle = \sqrt{\frac{8kT}{\pi m}}\)$
\textit{Среднеквадратичная скорость} - $$\sqrt{\frac{3kT}{m}}$$

\section{Барометрическая формула}
\textit{Выводится исходя из равновесия между силами тяжести и движением молекул воздуха. В результате получаем экспоненциальное законченное выражение для давления с высотой}
\textit{Итак, в МКТ барометрическая формула показывает экспоненциальное снижение давления с высотой, обусловленное тепловым движением молекул и гравитацией}
\begin{center}
    \big[\[ P(h) = P_0 \exp\left( - \frac{m g h}{k T} \right) \]\big]
    Где: - \(P(h)\) — давление на высоте \(h\), - \(P_0\) — давление на уровне моря, - \(m\) — масса одной молекулы воздуха, - \(g\) — ускорение свободного падения, - \(h\) — высота, - \(k\) — постоянная Больцмана, - \(T\) — абсолютная температура
\end{center}
\underline{Почему так:}

молекулы находятся в состоянии теплового равновесия, движутся хаотично, и их средняя кинетическая энергия связана с температурой \(T\). - Давление создаётся столкновениями молекул со стенками сосуда (или с окружающей средой). - В равновесии, при постоянной температуре и однородной массе молекул, давление убывает с высотой по экспоненциальному закону, так как молекулы теряют энергию, поднимаясь вверх против гравитации. Если масса молекулы \(m\) неизвестна, её можно выразить через молярную массу
\begin{center}
    \big[\(M\): \[ m = \frac{M}{N_A} \] где \(N_A\)\big]
    \newline Число Авогадро
    \[ P(h) = P_0 \exp\left( - \frac{M g h}{R T} \right) \] где \(R = N_A k\)
    \newline Газовая постоянная
\end{center}



\section{Распределение Больцмана}
\textit{Фундаментальное распределение вероятностей для частиц в термодинамической системе, которое описывает, как энергия или другие параметры (например, скорость) распределены между частицами при термостатическом равновесии}
\textit{Распределение Больцмана описывает статистические свойства систем в равновесии и является основой для понимания термодинамических закономерностей, таких как распределение энергий, скоростей и других параметров частиц}
\newline Основная формула распределения Больцмана для энергии:** Вероятность \(W(E)\) того, что система находится в состоянии с энергией \(E\)
\begin{center}
    \big[\[ W(E) = \frac{1}{Z} \exp\left( - \frac{E}{k T} \right) \]\big]
    \newline где: - \(E\) — энергия состояния, - \(k\) — постоянная Больцмана, - \(T\) — абсолютная температура, - \(Z\) — фактор разделения (или статистическая сумма), который обеспечивает нормировку вероятностей:
    \[ Z = \sum_{i} \exp \left( - \frac{E_i}{k T} \right) \] 
    \newline или в непрерывном случае
    \[ Z = \int \exp \left( - \frac{E}{k T} \right) g(E) \, dE \] где \(g(E)\)
    \newline Функция плостности состояний
\end{center}

\textbf{Распределение скоростей (распределение Максвелла-Больцмана)}

Для частиц с массой \(m\) распределение по скоростям даёт:
\begin{center}
    \big[\[ f(\mathbf{v}) = \left( \frac{m}{2 \pi k T} \right)^{3/2} \exp \left( - \frac{m v^2}{2 k T} \right) \]\big]
    \newline где: - \(\mathbf{v}\) — вектор скорости, - \(v = |\mathbf{v}|\)
\end{center}

Это распределение показывает, что большинство частиц имеют скорости около среднего значения, а вероятность наличия очень высоких скоростей уменьшается экспоненциально



\section{Центрифуга. Атомная программа}
\textbf{Центрифуга}
\newline — это устройство, которое использует центробежную силу для разделения смесей по плотности или массе. В МКТ при вращении частицы испытывают центробежную силу
\begin{center}
    \big[\(F_c = m \omega^2 r\), где \(m\)\big]
    масса, \(\omega\) — угловая скорость, а \(r\) — радиус вращения
\end{center}

\textit{В равновесии распределение частиц по радиусу или высоте в центрифуге можно описать с помощью распределения Больцмана, учитывая потенциальную энергию, связанную с центробежной силой.}

\textbf{Атомная программа в МКТ}
\newline концепция, что вся материя состоит из атомов и молекул, движущихся хаотично и взаимодействующих друг с другом. В рамках МКТ распределения скоростей и энергий частиц описываются законами Максвелла-Больцмана, что позволяет связывать микроскопические параметры с термодинамическими свойствами системы, такими как температура, давление и энергия.



\section{Энтропия. Физический смысл энтропии}
\textbf{Энтропия} - это термодинамическая величина, которая характеризует степень хаоса, беспорядка или неопределенности в системе.

\textbf{Физический смысл энтропии} - Мера беспорядка: Чем выше энтропия системы, тем более случайной и неупорядоченной она является.

\textbf{Количество возможных состояний}: Энтропия связана с числом микроскопических состояний системы, соответствующих её макроскопическому состоянию. Чем больше таких состояний, тем выше энтропия
\begin{center}
    \big[\[ S = k \ln \Omega \]\big]
    \newline где: - \(S\) — энтропия, - \(k\) — постоянная Больцмана, - \(\Omega\) — число возможных микроскопических состояний системы
\end{center}
\textbf{Термодинамическое направление процессов} -  В изолированной системе энтропия никогда не уменьшается; она либо остается постоянной (в обратимых процессах), либо увеличивается (в необратимых). Это формулировка второго закона термодинамики



\section{Второе начало термодинамики. Неравенство Клаузиуса}





\section{Тепловая машина. Вечный двигатель первого рода}

\section{Коэффициент полезного действия тепловой машины. Цикл Карно}
    

\end{document}
